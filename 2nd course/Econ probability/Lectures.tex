\documentclass[a4paper, 10pt]{article}
\usepackage{header}

\title{\LARGE{Теория вероятности и математическая статистика—1}\\
ФЭН НИУ ВШЭ}
\author{Винер Даниил  \href{https://t.me/danya_vin}{@danya\_vin}}
\date{Версия от \today}
\begin{document}
\maketitle
\tableofcontents
\newpage
\section{Дискретное вероятностное пространство. Базовые теоремы вероятности}
\definition $\Omega=\{\omega_1,\ldots,\omega_k,\ldots\}$ называется \textit{пространством элементарных исходов}, где $w_i$ — элементарный исход\\[2mm]
\definition $A$ — любое подмножество $\Omega$\\[2mm]
\definition Событие называется \textit{достоверным}, если $A=\Omega$\\[2mm]
\comment К $A$ применимы те же опреации, что используются с множествами:
\begin{itemize}
    \item $A\setminus B$ — вычитание
    \item $A\cap B$ — пересечение
    \item $A\cup B$ — объединение
    \item $A_c=\overline{A}=\Omega\setminus A$ — дополнение
\end{itemize}
\subsection{Полная группа несовместных событий}
\definition Это такой набор событий, для которого выполняются такие условия:
\begin{equation*}
    A_i\cap A_j=\varnothing\ \forall i\ne j
\end{equation*}
\begin{equation*}
    \displaystyle\bigcup\limits_i A_i=\Omega
\end{equation*}

\axiom $\forall \omega_i\ \exists p_i\geqslant0,$ при этом $\displaystyle\sum_i p_i=1$\\[2mm]
\corollary $0\leqslant p_i\leqslant1$\\[2mm]
\definition $P(A)=\sum_{w_i\in A} P(w_i)$, где $P(w_i)=p_i$\\[2mm]
\indent $(\Omega, P)$ — вероятностное пространство в дискретном случае
\subsection{Подходы к определению вероятностей}
\begin{enumerate}
    \item Априорный (предварительное знание)
    \item Частотный (предел ряда частот)
    \item Модельный (математическая модель)
\end{enumerate}
\subsection{Классическое определение вероятности}
\indent Имеет место, когда исходы равновероятны\\[2mm]
\definition
$$P(A)=\displaystyle\frac{|A|}{|\Omega|}=\displaystyle\frac{\text{число благоприятных исходов}}{\text{количество всех исходов}}$$
\subsection{Теорема сложения}
\indent $P(A\cup B)=P(A)+P(B)-P(A\cap B)$\\[2mm]
\indent $P(A_1\cup\ldots\cup A_n)=\sum_{i=1}^{n} P(A_i)-\sum_{i<j} (A_i\cap A_j)+\ldots+(-1)^{n-1}P(A_1\cap\ldots\cap A_n)$
\subsection{Условная вероятность}
$P(A|B)=\displaystyle\frac{P(A\cap B)}{P(B)}\ \forall B: P(B)>0$
\subsection{Теорема умножения}
$P(A_1\cap\ldots\cap A_n)=P(A_1)\cdot P(A_2|A_1)\cdot P(A_3|A_1\cap A_2)\cdot\ldots\cdot P(A_n|A_1\cdot\ldots\cdot A_{n-1})$



\end{document}
