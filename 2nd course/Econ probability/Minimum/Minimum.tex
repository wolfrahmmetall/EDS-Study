\documentclass{article}
\usepackage{header} % Required for inserting images
% \newcommand{\p}[1]{$\mathbb{P}(#1)$}

\title{\LARGE{Теория вероятности и математическая статистика—1}\\
Теоретический и задачный минимумы\\
ФЭН НИУ ВШЭ}
\author{Винер Даниил  \href{https://t.me/danya_vin}{@danya\_vin}}
\date{Версия от \today}

\begin{document}
\maketitle
\tableofcontents
\newpage
\setlength{\parindent}{15pt}
\setlength{\parskip}{2mm}
\section{Теоретический минимум}
\subsection{Сформулируйте классическое определение вероятности}
Имеет место, когда исходы равновероятны

\definition
$$P(A)=\displaystyle\frac{|A|}{|\Omega|}$$

\definition $P(A)=\sum_{\omega_i\in A}p(\omega_i)$
\subsection{Выпишите формулу условной вероятности}
$P(A|B)=\displaystyle\frac{P(A\cap B)}{P(B)}\ \forall B: P(B)>0$

\subsection{Дайте определение независимости (попарной и в совокупности) для $n$ случайных событий}
\definition События $A\text{ и }B$ называются \textbf{попарно независимыми}, если:
\begin{equation*}
    \begin{aligned}
        P(A\cap B)&=P(A)\cdot P(B)\\
        P(A|B)P(B)&=P(A)\cdot P(B)\text{ — вытекает интуитивное определение}
    \end{aligned}
\end{equation*}

\definition События $A_1,\ldots, A_n$ независимы в совокупности, если:
\begin{equation*}
    \begin{aligned}
    \forall i_1<\ldots<i_k<\ldots<i_n\ \forall k=1,\ldots,n:\\ P(A_{i_1}\cap A_{i_2}\cap\ldots\cap A_{i_k})=P(A_{i_1})\cdot P(A_{i_2})\cdot \ldots\cdot P(A_{i_k})
    \end{aligned}
\end{equation*}

\comment  Для $A_1,A_2,A_3$:
\begin{equation*}
    \begin{aligned}
        P(A_1\cap A_2)&=P(A_1)\cdot P(A_2)\\
        P(A_2\cap A_3)&=P(A_2)\cdot P(A_3)\\
        P(A_1\cap A_3)&=P(A_1)\cdot P(A_3)\\
        P(A_1\cap A_2\cap A_3)&=P(A_1)\cdot P(A_2)\cdot P(A_3)
    \end{aligned}
\end{equation*}

\subsection{Выпишите формулу полной вероятности, указав условия её применимости}
Пусть $\{H_i\}$ — полная группа несовместных событий (разбиение $\Omega$)

Должны быть выполнены такие свойства:
\begin{itemize}
    \item $H_i\cap H_j=\varnothing\ \forall i\ne j$ — несовместность\\[1mm]
    \item $\displaystyle\bigcup_{i=1}^{n}H_i=\Omega$ — полнота
\end{itemize}

\theorem Тогда, $P(A)=\sum_{i=1}^{n}P(A|H_i)P(H_i)$

\proof
\begin{equation*}
    \begin{aligned}
        P(A)&=P\left(\bigcup_{i=1}^{n}(A\cap H_i)\right)\\
        &=\sum_{i=1}^{n}P(A\cap H_i)\\
        &=\sum_{i=1}^{n} P(A|H_i)\cdot P(H_i)
    \end{aligned}
\end{equation*}\qed

\subsection{Выпишите формулу Байеса, указав условия её применимости}
Пусть $H_1, H_2, \ldots$ — полная группа собътий, и $A$ — некоторое собътие, вероятность которого положительна. Тогда условная вероятность того, что имело место событие $H_k$, если в резулътате эксперимента наблюдалосъ событие $A$, может быть вычислена по формуле
\begin{equation*}
    \begin{aligned}
        P(H_k|A)&=\frac{P(A|H_k)\cdot P(H_k)}{P(A)}\\
        &=\frac{P(H_k\cap A)}{P(A)}\\
        &=\frac{P(A|H_k)\cdot P(H_k)}{\sum_{i=1}^{n} P(A|H_i)P(H_i)}
    \end{aligned}
\end{equation*}

\newpage
\section{Задачный минимум}
\subsection{$P(A)=0.3,P(B)=0.4,P(A\cap B)=0.1$}
\begin{enumerate}
    \item[\textbf{a)}] Найдите $P(A|B)$
    
    $P(A|B)=\displaystyle\frac{P(A\cap B)}{P(B)}=\frac{0.1}{0.4}=0.25$
    \item[\textbf{b)}] Найдите $P(A\cup B)$
    
    $P(A\cup B)=P(A)+P(B)-P(A\cap B)=0.3+0.4-0.1=0.6$
    \item[\textbf{c)}] Являются ли события $A$ и $B$ независимыми?

    \definition События $A$ и $B$ называются независимыми, если $P(A\cap B)=P(A)\cdot P(B)$

    \definition События $A$ и $B$ называются несовместными, если $A\cap B=\varnothing$

    Пусть $P(A)\ne0,P(B)\ne0$. Тогда, $A$ и $B$ несовместны, то $A$ и $B$ зависимы

    $$0=P(A\cap B)=P(A)\cdot P(B)\ne0\Longrightarrow A \text{ и } B \text{ зависимы}$$
\end{enumerate}

\subsection{Карлсон выложил кубиками слово КОМБИНАТОРИКА...}
\subsubsection*{Способ №1 (С помощью формулы умножения вероятностей)}
$P(A_1\cap \ldots \cap A_n)=P(A_1)\cdot P(A_2|A_1)\cdot P(A_3|A_1\cap A_2)\cdot\ldots\cdot P(A_n|A_1\cap\ldots\cap A_{n-1})$



Пусть имеются такие события: \begin{equation*}
    \begin{aligned}
        A_1&:=\{\text{первая буква — К}\}\\
        A_2&:=\{\text{вторая буква — О}\}\\
        A_3&:=\{\text{третья буква — Р}\}\\
        A_4&:=\{\text{четвертая буква — Т}\}
    \end{aligned}
\end{equation*}

Тогда, искомая вероятность:
\begin{equation*}
\begin{aligned}
    P(A_1\cap A_2\cap A_3\cap A_4)&=P(A_1)\cdot P(A_2|A_1)\cdot P(A_3|A_1\cap A_2)\cdot P(A_4|A_1\cap A_2\cap A_3)\\
    &=\frac{2}{13}\cdot\frac{2}{12}\cdot\frac{1}{11}\cdot\frac{1}{10}\\
    &=\frac{1}{4290}
    \end{aligned}
\end{equation*}

\subsubsection*{Способ №2 (комбинаторный)}
$P(A)=\displaystyle\frac{|A|}{|\Omega|},\ \Omega=\{(a_1,a_2,a_3,a_4):a_1\in L, a_2\in L, a_3\in L, a_4\in L, a_i\ne a_j\text{ при }i\ne j\}$

$|\Omega|=\displaystyle\frac{13!}{9!}=17160$

$A=\{(K_1,O_1,P_1,T_1),(K_2,O_1,P_1,T_1),(K_1,O_2,P_1,T_1),(K_2,O_2,P_1,T_1)\}\longrightarrow$ 4 исхода

Индекс у букв означают какой по счету встретилась буква в слове <<КОМБИНАТОРИКА>>

Тогда, искомая вероятность$=\displaystyle\frac{|A|}{|\Omega|}=\frac{4}{17160}=\frac{1}{4290}$

\subsection{В первой урне 7 белых и 3 черных шара, во второй — 8 белых и 4 черных шара, в третьей — 2 белых и 13 черных шаров}
$D_i:=\{\text{выбираем $i$-ю урну}\}$, где $i=1,2,3$ — разбиение $\Omega$

Заметим, что урну мы выбираем равновероятно, то есть $P(D_1)=P(D_2)=P(D_3)=\displaystyle\frac{1}{3}$
\begin{enumerate}
    \item[\textbf{a)}] Вычислите вероятность того, что шар, взятый наугад из выбранной урны, окажется белым
    
    \textbf{Формуа полной вероятности}
    $$P(A)=P(A|D_1)\cdot P(D_1)+\ldots+P(A|D_n)\cdot P(D_n)$$
    В нашем случае, формула будет иметь вид
    $$P(A)=P(A|D_1)\cdot P(D_1)+P(A|D_2)\cdot P(D_2)+P(A|D_3)\cdot P(D_3)$$
    $A:=\{\text{шар оказался белым}\}$

    Заметим, что $P(A|D_1)=\frac{7}{10},P(A|D_2)=\frac{2}{3},P(A|D_3)=\frac{2}{15}$, тогда
    \begin{equation*}
        \begin{aligned}
            P(A)&=P(A|D_1)\cdot P(D_1)+P(A|D_2)\cdot P(D_2)+P(A|D_3)\cdot P(D_3)\\
            &=\frac{7}{10}\cdot\frac{1}{3}+\frac{2}{3}\cdot\frac{1}{3}+\frac{2}{15}\cdot\frac{1}{3}\\
            &=\frac{1}{2}
        \end{aligned}
    \end{equation*}
    \item[\textbf{b)}] $P(D_1|A)=\displaystyle\frac{P(A|D_1)\cdot P(D_1)}{P(A|D_1)P(D_1)+P(A|D_2)P(D_2)+P(A|D_3)P(D_3)}=\frac{7}{15}$
\end{enumerate}

\subsection{В операционном отделе банка работает 80\% опытных сотрудников и 20\% неопытных}
Обозначим сотрудников так:
\begin{equation*}
    \begin{aligned}
        D_1&:=\{\text{\textbf{опытный} сотрудник}\}\\
        D_2&:=\{\text{\textbf{неопытный} сотрудник}\}
    \end{aligned}
\end{equation*}

Пусть $A:=\{\text{совершена ошибка}\}$

Тогда, условия задачи можно записать так:
\begin{equation*}
    \begin{aligned}
        P(A|D_1)&=0.01\\
        P(A|D_2)&=0.1
    \end{aligned}
\end{equation*}

\begin{enumerate}
    \item[\textbf{a)}] $P(A)=P(A|D_1)\cdot P(D_1)+P(A|D_2)\cdot P(D_2)=0.01\cdot0.8+0.1\cdot0.2=0.028$
    \item[\textbf{b)}] $P(D_2|A)=\displaystyle\frac{P(A|D_2)\cdot P(D_2)}{P(A)}=0.714$

    Заметим, что события $(D_2|A)$ и $(D_1|A)$ образуют полную группу вероятностей, то есть $$P(D_2|A)+P(D_1|A)=1\Longrightarrow P(D_1|A)=0.286$$
\end{enumerate}

\end{document}
