\documentclass[a4paper, 10pt]{article}
\usepackage{standart-header}
\usepackage{textcomp}
\fancyhead[R]{Винер Даниил. БЭАД232}
\fancyfoot [C] {\thepage}
\usepackage{pgfplots}
\setlength{\parindent}{15pt}
\setlength{\parskip}{2mm}
% \setlist[itemize]{left=1cm}
% \setlist[enumerate]{left=1cm}
\fancyhead[L]{ТВиМС—1. Семинар 04.10.2024}

\begin{document}
\definition Пусть задано измеримое пространство $(\Omega,\fk)$. Функция $\xi:\Omega\longrightarrow\mathbb{R}$, $\xi=\xi(\omega)$, $\omega\in\Omega$ называется \textit{измеримой} функцией относительно \s-алгебры $\fk$, если $$\forall c\in\mathbb{R}\ \{\omega\in\Omega:\xi(\omega)>c\}\in\fk$$

Назовем условием измеримости функцию, измеримую относительно \s-алгебры $\fk$, также называемую $\fk$-измеримой функцией или функцией, согласованной с \s-алгеброй $\fk$
    
% В теории вероятностей $\fk$—измеримая функция называется случайными мвеличинами (или $\fk$-измеримой с.в.)



\section*{Задача №1}
$\Omega=\{\heartsuit, \diamondsuit,\spadesuit,\clubsuit\}$, $\fk=\{\varnothing, \{\heartsuit,\diamondsuit\}, \{\spadesuit\}, \{\clubsuit\}, \{\heartsuit,\diamondsuit,\spadesuit\}, \{\heartsuit,\diamondsuit,\clubsuit\}, \{\spadesuit, \clubsuit\}, \Omega\}$, $\xi:\Omega\rightarrow\mathbb{R}, \eta:\Omega\rightarrow\mathbb{R}$

\begin{table}[h]
    \begin{tabular}{|c|c|c|c|c|}
        \hline
        $\omega$ & $\heartsuit$ & $\diamondsuit$ & $\spadesuit$  & $\clubsuit$\\
        \hline
        $\xi(\omega)$ & 1 & 1 & 2 & 3\\
        \hline
        $\eta(\omega)$ & 3 & 2 & 1 & 1\\
        \hline
        
    \end{tabular}
\end{table}
\begin{itemize}
    \item[a)] Является ли $\xi$ $\fk$-измеримой? мяу мяу мяу всем привет от яны
    \begin{itemize}
        \item[1)] $c>3:\ \{\omega\in\Omega:\ \xi(\omega)>c=100\}=\varnothing\in\fk$
        \item[2)] $c=3:\ \{\omega\in\Omega:\ \xi(\omega)>3\}=\varnothing\in\fk$
        \item[3)] $c\in(2;3):\ \{\omega\in\Omega:\ \xi(\omega)>c=27\}=\{\clubsuit\}\in\fk$ 
        \item[4)] $c=2:\ \{\omega\in\Omega:\ \xi(\omega)>2\}=\{\clubsuit\}\in\fk$  
        \item[5)] $c\in(1;2):\ \{\omega\in\Omega:\ \xi(\omega)>c=13\}=\{\spadesuit,\clubsuit\}\in\fk$
        \item[6)] $c=1:\ \{\omega\in\Omega:\ \xi(\omega)>1\}=\{\spadesuit,\clubsuit\}\in\fk$  
        \item[7)] $c<1:\ \{\omega\in\Omega:\ \xi(\omega)>c=-200\}=\Omega\in\fk$ 
    \end{itemize}
    \item[б)] Является ли $\eta$ $\fk$-измеримой?
    
    $c=2.5:\ \{\omega\in\Omega:\eta(\omega)>2.5\}=\{\heartsuit\}\notin\fk\Longrightarrow$ не является $fk$-измеримой 
\end{itemize}

\section*{№2}
$\Omega=\mathbb{R},\ \fk=\mathcal{B}(\mathbb{R}),\xi(\omega)=\omega^2$. Определить, $\xi$ — $\fk$-измерима?

\begin{itemize}
    \item $c<0:\ \{\omega\in\Omega:\xi(\omega)>c\}=\{\omega\in\mathbb{R}:\underbrace{\omega^2}_{\geqslant0}>c\}=\mathbb{R}\in\bk(\mathbb{R})$
    \item $c\geqslant0:$
    \begin{equation*}
        \begin{aligned}
            \{\omega\in\Omega:\xi(\omega)>c\}&=\{\omega\in\mathbb{R}:\omega^2>c\}\\
            &=\{\omega\in\mathbb{R}:\omega<-\sqrt{c}\}\cup\{\omega\in\mathbb{R}:\omega>\sqrt{c}\}\\
            &=\left(-\infty;-\sqrt{c}\right)\cup\left(\sqrt{c};+\infty\right)\\
            &=\underbrace{\mathbb{R}}_{\bk(\mathbb{R})}\setminus\underbrace{[-\sqrt{c};\sqrt{c}]}_{\in\bk(\mathbb{R})}\in\bk(\mathbb{R})\in\fk
        \end{aligned}
    \end{equation*}
\end{itemize}


\section*{№3}
$(\Omega, \fk)$ — измеримое пространство, $\xi:\Omega\rightarrow\mathbb{R}$ — $\fk$-измеримая функция $\forall c\in\mathbb{R}$

Доказать утверждения
\begin{itemize}
    \item[\textbf{а)}] $\underbrace{\{\omega\in\Omega:\ \xi(\omega)\geqslant c\}}_{=LHS}=\underbrace{\displaystyle\bigcap_{i=1}^n \left\{\omega\in\Omega:\ \xi(\omega)>c-\frac{1}{n}\right\}\in\fk}_{=RHS}$
    \begin{itemize}
        \item $(LHS\subseteq RHS)$: $\omega_0\in LHS\Longrightarrow\xi(\omega)\geqslant c\Longrightarrow\forall n\in\mathbb{N}:  \xi{w_0}>c-\frac{1}{n}\Longrightarrow \omega_0\in RHS$
        \item $(RHS\subseteq LHS)$: $\omega_0\in RHS\Longrightarrow\forall n\in\mathbb{N}: \xi(\omega_0)>c-\frac{1}{n}\Longrightarrow\xi(\omega_0)\geqslant\lim\limits_{n\mapsto\infty}\left(c-\frac{1}{n}\right)=c\Longrightarrow\omega_0\in LHS$
    \end{itemize}
    \item[\textbf{b)}] $\{\omega\in\Omega:\xi(\omega)=c\}=\underbrace{\{\omega\in\Omega:\xi(\omega)\geqslant c\}}_{\in\fk\text{(см.п.а)}}\setminus\underbrace{\{\omega\in\Omega:\xi(\omega)>c\}}_{\in\fk\text{(по опр. $\fk$-изм.ф-и)}}\in\fk$
    \item[\textbf{c)}] $\{\omega\in\Omega:\xi(\omega)\leqslant c\}=\underbrace{\Omega}_{\in\fk}\setminus\underbrace{\{\omega\in\Omega:\xi(\omega)>c\}}_{\in\fk\text{(по опр. $\fk$-изм.ф-и)}}\in\fk$ 
    \item[\textbf{d)}] $\{\omega\in\omega:\xi(\omega)<c\}=\underbrace{\{\omega\in\Omega:\xi(\omega)\leqslant c\}}_{\in\fk\text{(см. п.c)}}\setminus\underbrace{\{\omega\in\Omega:\xi(\omega)=c\}}_{\in\fk\text{(см.п.b)}}$
\end{itemize}

\section*{№4}
$(\Omega,\fk)$ — измеримое пространство, $\xi:\Omega\rightarrow\mathbb{R}$ — $\fk$-измеримая функция

Докажите, что $\xi^2(\omega)$ — $\fk$-измерима
\begin{itemize}
    \item $c<0:\ \{\omega\in\Omega:\xi^2(\omega)>c\}=\Omega\in\fk$
    \item $c\geqslant 0:\ \{\omega\in\Omega:\xi^2(\omega)>c\}=\underbrace{\{\omega\in\Omega:\xi(\omega)<-\sqrt{c}\}}_{\in\fk\text{(см. п.3d)}}\cup\underbrace{\{\omega\in\Omega:\xi(\omega)>\sqrt{c}\}}_{\in\fk\text{(по опр.$\fk$-изм.ф-и)}}\in\fk$
\end{itemize}

\end{document}