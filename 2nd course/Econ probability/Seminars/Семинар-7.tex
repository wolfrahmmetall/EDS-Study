\documentclass[a4paper, 10pt]{article}
\usepackage{standart-header}
\usepackage{textcomp}
\fancyhead[R]{Винер Даниил. БЭАД232}
\fancyfoot [C] {\thepage}
\usepackage{pgfplots}
\setlength{\parindent}{15pt}
\setlength{\parskip}{2mm}
% \setlist[itemize]{left=1cm}
% \setlist[enumerate]{left=1cm}
\fancyhead[L]{ТВиМС—1. Семинар 18.10.2024}

\begin{document}
\definition Случайная величина $\xi$ абсолютно непрерывная если существует интегрируемая функция $f_{\xi}(t)$, такая что для любого $x\in\mathbb{R}$:
\begin{equation*}
    F_{\xi}=\int_{-\infty}^{x}f_{\xi}(t)dt
\end{equation*}
При этом, функция $f_{\xi}(t)$ называется плотностью распределения случайной величины $\xi$

\section*{№1}
$\Omega=[0;1]$, $\fk=\bk([0;1])$, $\prob{A}$ — длина $A\in\fk$, $\xi(\omega)=\omega^2$ — случайная величина

\subsection*{a)}
\begin{equation*}
    \begin{aligned}
        F_{\xi}(x)&=\prob{\xi\leqslant x}\\
        &=\prob{\omega\in\Omega:\xi(\omega)\leqslant x}\\
        &=\prob{\omega\in[0;1]:\omega^2\leqslant x}\\
        &=\begin{cases}
            1,&x<0\\
            \sqrt{x},&x\in[0;1]\\
            1,&x>1
        \end{cases}
    \end{aligned}
\end{equation*}

Рассмотрим $x\in[0;1]:$
\begin{equation*}
    \begin{aligned}
        \prob{\omega\in[0;1]:\omega^2\leqslant x}=\prob{\omega\in[0;1]:-\sqrt{x}\leqslant \omega\leqslant\sqrt{x}}\Longrightarrow\prob{[0;\sqrt{x}]}
    \end{aligned}
\end{equation*}


\subsection*{b)}
$f_{\xi}(x)=\displaystyle\frac{d}{dx}F_{\xi}(x)=\begin{cases}
    0,&x<0\\
    \frac{1}{2\sqrt{x}},&x\in[0;1]\\
    0,&x>1
\end{cases}$

\subsection*{c)}
\begin{equation*}
    \begin{aligned}
        \matwait{\xi}&=\int_{-\infty}^{+\infty}xf_{\xi}(x)dx\\
        &=\int_0^1 x\frac{1}{2\sqrt{x}}dx\\
        &=\frac{1}{2}\int_0^1\sqrt{x}dx\\
        &=\frac{1}{2}\cdot\left.\frac{x^{\frac{3}{2}}}{\frac{3}{2}}\right\vert_{x=0}^{x=1}\\
        &=\frac{1}{3}
    \end{aligned}
\end{equation*}

\subsection*{d)}
\begin{equation*}
    \begin{aligned}
        \matwait{\xi^2}&=\int_{-\infty}^{+\infty} x^2f_{\xi}(x)dx\\
        &=\int_0^1 x^2\frac{1}{2\sqrt{x}}dx\\
        &=\frac{1}{2}\cdot\left.\frac{x^{\frac{5}{2}}}{\frac{5}{2}}\right\vert_{0}^1\\
        &=\frac{1}{5}
    \end{aligned}
\end{equation*}

\subsection*{e)}
\begin{equation*}
    \begin{aligned}
        \dispersia{\xi}&=\matwait{\xi^2}-(\matwait{\xi})^2\\
        &=\frac{1}{5}-\left(\frac{1}{3}\right)^2\\
        &=\frac{4}{45}
    \end{aligned}
\end{equation*}


\section*{№2}
$f_{\xi}(x)=\begin{cases}
    cx,&x\in[0;1]\\
    0,&x\notin[0;1]
\end{cases}$

\subsection*{a)}
$\lim\limits_{x\longrightarrow+\infty}F_{\xi}(x)=1$:
\begin{equation*}
    \begin{aligned}
        1&=\int_{-\infty}^{+\infty} f_{\xi}(t)dt\\
        &=\int_0^1 ctdt\\
        &=c\left.\frac{t}{2}\right\vert_0^1\\
        &=\frac{c}{2}
    \end{aligned}\Longrightarrow c=2
\end{equation*} 

\subsection*{b)}
\definition Пусть $\xi$ — абсолютная непрерывная величина. Тогда, $\forall B(\bk(\mathbb{R}))$:
\begin{equation*}
    \prob{\xi\in B}=\int\limits_{B}f_{\xi}(t)dt
\end{equation*}

Требуется найти $\prob{\xi\leqslant\frac{1}{2}}$. Тогда, исходя из определения выше:
\begin{equation*}
    \begin{aligned}
        \int_{0}^{0.5} f_{\xi}(t)dt&=\int_0^{0.5}2tdt\\
        &=\left.t^2\right\vert_{0}^{0.5}\\
        &=\frac{1}{4}
    \end{aligned}
\end{equation*}


\subsection*{c)}
\begin{equation*}
    \begin{aligned}
        \prob{\xi\in[0.5;1.5]}&=\displaystyle\int\limits_{B}f_{\xi}(t)dt\\
        &=\int_{0.5}^1 2tdt\\
        &=\left.t^2\right\vert_{0.5}^1\\
        &=\frac{3}{4}
    \end{aligned}
\end{equation*}


\subsection*{d)}
$\prob{\xi\in[2;3]}=\int\limits_{[2;3]}\underbrace{f_{\xi}(t)}_{=0,\text{по усл.}}dt=0$


\subsection*{e)}
Пусть $x<0\Longrightarrow F_{\xi}(x)=\displaystyle\int_{-\infty}^{\infty}f_{\xi}(t)dt=0$

Пусть $0\leqslant x\leqslant 0\Longrightarrow$
\begin{equation*}
    \begin{aligned}
        F_{\xi}(x)&=\displaystyle\int_{-\infty}^{\infty}f_{\xi}(t)dt\\
        &=\int_{-\infty}^0 f_{\xi}(t)dt+\int_{0}^{x}\\
        &=\left.t^2\right\vert_0^x\\
        &=x^2
    \end{aligned}
\end{equation*}

Пусть $x>1$, тогда
\begin{equation*}
    \begin{aligned}
        F_{\xi}(x)&=\int_{-\infty}^{x} f_{\xi}(t)dt\\
        &=\int_{-\infty}^{0}f_{\xi}(t)dt+\int_0^1 f_{\xi}(t)dt+\int_{1}^x f_{\xi}(t)dt\\
        &=1
    \end{aligned}
\end{equation*}

\subsection*{f)}
\begin{equation*}
    \begin{aligned}
        \matwait{\xi}&=\int_{-\infty}^{\infty} xf_{\xi}(x)dx\\
        &=\int_{0}^{1} x2xdx\\
        &=2\int_{0}^{1} x^2dx\\
        &=\frac{2}{3}
    \end{aligned}
\end{equation*}

\subsection*{g)}
\begin{equation*}
    \begin{aligned}
        \matwait{\xi^2}&=\int_{-\infty}^{\infty} x^2f_{\xi}(x)dx\\
        &=\int_{0}^{1} x^22xdx\\
        &=2\int_{0}^{1} x^3dx\\
        &=\frac{1}{2}
    \end{aligned}
\end{equation*}

\subsection*{h)}
\begin{equation*}
    \begin{aligned}
        \dispersia{\xi}&=\matwait{\xi^2}-\left(\matwait{\xi}\right)^2\\
        &=\frac{1}{2}-\frac{4}{9}\\
        &=\frac{1}{18}
    \end{aligned}
\end{equation*}
\subsection*{i)}
\begin{equation*}
    \begin{aligned}
        \matwait{\sqrt{\xi}}&=\int_{0}^1 \sqrt{x}2xdx\\
        &=2\int_0^1x^{\frac{3}{2}}dx\\
        &=2\cdot\left.\frac{x^{\frac{5}{2}}}{\frac{5}{2}}\right\vert_{x=0}^{x=1}\\
        &=\frac{4}{5}
    \end{aligned}
\end{equation*}

\definition Пусть случайная величина $\xi$ имеет плотность $f_{\xi}(t)$

\textit{Квантилью} уровня $u\in(0;1)$ случайной величины $\xi$ называется наименьшее число $q\in\mathbb{}R$:
\begin{equation*}
    \int_{-\infty}^q f_{\xi}(t)dt=u
\end{equation*}

\section*{№3}
$\xi$ — продолжительность ругания Васи, $\xi\sim\text{Exp}(\lambda)\Longleftrightarrow f_{\xi}(x)=\begin{cases}
    0,&x <0\\
    \lambda e^{-\lambda x},&x\geqslant 0
\end{cases}$

$\matwait{\xi}=\displaystyle\frac{1}{\lambda}=6\Longrightarrow\lambda=6,\ \dispersia{\xi}=\frac{1}{\lambda^2}$

$F_{\xi}(x)=\begin{cases}
    0,&x<0\\
    1-e^{-\lambda x},&x\geqslant0
\end{cases}$

\subsection*{a)}
\begin{equation*}
    \begin{aligned}
        \prob{\xi>6}&=1-\prob{\xi\leqslant 6}\\
        &=1-F_{\xi}(6)\\
        &=1-\left[1-e^{-\lambda6}\right]\\
        &=e^{-\lambda6}\\
        &=e^{-1}
    \end{aligned}
\end{equation*}

\subsection*{b)}
$\dispersia{\xi}=\displaystyle\frac{1}{\lambda^2}=36$

\subsection*{c)}
\begin{equation*}
    \begin{aligned}
        \mathbb{P}(\{\xi\leqslant 7\}|\{\xi>6\})&=\frac{\prob{6<\xi\leqslant 7}}{\underbrace{\prob{\xi>6}}_{=e^{-1}}}\\
        &=\frac{F_{\xi}(7)-F_{\xi}(6)}{e^{-1}}\\
        &=1-e^{\frac{1}{6}}
    \end{aligned}
\end{equation*}


\end{document}