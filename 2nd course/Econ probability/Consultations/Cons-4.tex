\documentclass[a4paper, 10pt]{article}
\usepackage{standart-header}
\fancyhead[R]{Винер Даниил. БЭАД232}
\fancyfoot [C] {\thepage}
\setlength{\parindent}{15pt}
\setlength{\parskip}{2mm}
\newcommand{\orr}{\text{ or }}
\newcommand{\andd}{\text{ and }}
\newcommand{\prob}[1]{\mathbb{P}\left(\left\{#1\right\}\right)}
\newcommand{\matwait}[1]{\mathbb{E}\left[#1\right]}
\newcommand{\dispersia}[1]{\mathbb{D}\left[#1\right]}
\fancyhead[L]{ТВиМС—1. Консультация—1 по максимуму 18.10.2024}

\begin{document}
\section*{№4}
Дано: $\xi$ — абсолютная непрерывная величина, $\eta=2\xi+7$

Найти: $F_{\eta}(y),f_{\eta}(y)$

\begin{enumerate}
    \item Найдем $F_{\eta}(y)$
    \begin{equation*}
        \begin{aligned}
            F_{\eta}(y)&=\prob{\eta\leqslant y}\\
            &=\prob{2\xi+7\leqslant y}\\
            &=\prob{\xi\leqslant\frac{y-7}{2}}\\
            &=F_{\xi}(y)
        \end{aligned}
    \end{equation*}
    \item Теперь продифференцируем функцию распределения и найдем функцию плотности
    \begin{equation*}
        \begin{aligned}
            f_{\eta}(y)&=\frac{d}{dy}F_{\eta}(y)\\
            &=\frac{d}{dy}F_{\xi}\left(\frac{y-7}{2}\right)\\
            &=f_{\xi}\left(\frac{y-7}{2}\right)\cdot\frac{1}{2}
        \end{aligned}
    \end{equation*}
\end{enumerate}


\section*{№5}
Дано: $\xi$ — абсолютная непрерывная величина, $\eta=8-9\xi$

Найти: $F_{\eta}(y),f_{\eta}(y)$

\comment Так как $\xi$ — абсолютная непрерывная величина, то $\prob\xi\in B=\displaystyle\int\limits_{B}f_{\xi}(t)dt$. Тогда, можно записать, что 
\begin{equation*}
    \begin{aligned}
        \prob{\xi=a}&=\prob{\xi\in[a;a]}\\
        &=\int_a^af_{\xi}(t)dt\\
        &=0
    \end{aligned}
\end{equation*}

Найдем $F_{\eta}(y):$
\begin{equation*}
    \begin{aligned}
        F_{\eta}(y)&=\prob{\eta\leqslant y}\\
        &=\prob{8-9\xi\leqslant y}\\
        &=\prob{\xi\geqslant \frac{8-y}{9}}\\
        &=1-\prob{\xi<\frac{8-y}{9}}\\
        &=1-\prob{\xi\leqslant\frac{8-y}{9}}\\
        &=1-F_{\xi}\left(\frac{8-y}{9}\right)
    \end{aligned}
\end{equation*}

Тогда, $f_{\eta}(y)=f_{\xi}\left(\displaystyle\frac{8-y}{9}\right)\cdot\displaystyle\frac{1}{9}$

\definition Говорят, что случайная величина $\xi$ имеет равномерное распределение на отрезке $[a;b]$, $\xi\sim U(a;b)$, если $\xi$ имеет плотность вида $f_{\xi}(t)=\begin{cases}
    \frac{1}{b-a},&\text{ если }a\leqslant t\leqslant b\\
    0,&\text{ иначе}
\end{cases}$

$F_{\xi}(x)=\begin{cases}
    0,&x<a\\
    \frac{x-a}{b-a},&a\leqslant x\leqslant b\\
    1,&x>b
\end{cases}$

$\matwait{\xi}=\displaystyle\frac{a+b}{2},\ \dispersia{\xi}=\displaystyle\frac{(b-a)^2}{12}$

\section*{№6}
Дано: $\xi\sim U(0;1)$, $\eta=1-2\xi$
\subsection*{a)}
В данной ситуации, $F_{\xi}(x)=\begin{cases}
    0,&x<0\\
    x,&x\in[0;1]\\
    1,&x>1
\end{cases}$

Тогда,
\begin{equation*}
    \begin{aligned}
        F_{\eta}(y)&=\prob{\eta\leqslant y}\\
        &=\prob{1-2\xi\leqslant y}\\
        &=\prob{\xi\geqslant\frac{1-y}{2}}\\
        &=1-\prob{\xi<\frac{1-y}{2}}\\
        &=1-\prob{\xi\leqslant\frac{1-y}{2}}\\
        &=1-F_{\eta}\left(\frac{1-y}{2}\right)\\
        &=1-\begin{cases}
            0,&\frac{1-y}{2}<0\\
            \frac{1-y}{2},&0\leqslant\frac{1-y}{2}\leqslant1\\
            1,&\frac{1-y}{2}>1
        \end{cases}\\
        &=1-\begin{cases}
            0,&y>1\\
            \frac{1-y}{2},&y\in[-1;1]\\
            1,&y<-1
        \end{cases}\\
        &=\begin{cases}
            1,&y>1\\
            \frac{1+y}{2},&y\in[-1;1]\\
            0,&y<-1
        \end{cases}
    \end{aligned}
\end{equation*}

\subsection*{b)}
$f_{\eta}(y)=\displaystyle\frac{d}{dy}F_{\eta}(y)=\begin{cases}
    0,&y<-1\\
    \frac{1}{2},&y\in[-1;1]\\
    0,&y>1
\end{cases}$

Получается, что $\eta\sim U(0;1)$

\textbf{Вывод:} если какая-то случайная величина $\xi$ имеет равномерное распределение, а величина $\eta$ зависит от $\xi$, то $\eta$ также имеет равномерное распределение

\subsection*{c)}
\begin{equation*}
    \begin{aligned}
        \prob{\eta\in[-0.5;0.5]}&=\int_{-0.5}^{0.5}f_{\eta}(y)dy\\
        &=\int_{-0.5}^{0.5}\frac{1}{2}dy\\
        &=\frac{1}{2}\cdot1\\
        &=\frac{1}{2}
    \end{aligned}
\end{equation*}

\subsection*{d)}
трубется найти медиану $M$
\begin{equation*}
    \begin{cases}
        \int\limits_{-\infty}^{M} f_{\xi}(y)dy = 0.5\\
        \int \limits_{-1}^{M} \frac{1}{2} dy = 0.5
    \end{cases}\Longrightarrow\frac{1}{2}(M+1)=0.5\Longrightarrow M=0
\end{equation*}

\section*{№7}
$f_{\xi} =\begin{cases}
    0,&x > 1\\
    \frac 3 {x^4},& x \geqslant 1
\end{cases},\ \eta = \ln {\xi}$
    
Пусть $x<1$, тогда $F_{\xi}(x)=\displaystyle\int_{-\infty}^{x} f_{\xi}(t)dt\Longrightarrow F_{\xi}(x)=0$

Пусть $x\geqslant1$, тогда 
\begin{equation*}
    \begin{aligned}
        F_{\xi}(x)&=\int_{-\infty}^{1}f_{\xi}(t)dt+\int_1^x f_{\xi}(t)dt\\
        &=3\cdot\frac{t^{-3}}{-3}\vert^{t=x}_{t=1}\\
        &=1-\frac{1}{x^3}
    \end{aligned}
\end{equation*}
Тогда, $F_{\xi}=\begin{cases}
    0,&x<1\\
    1-\frac{1}{x^3},&x\geqslant 1
\end{cases}$

\subsection*{a) Найти $F_{\xi}(y)$}
\begin{equation*}
    \begin{aligned}
        F_{\eta}(y)&=\prob{\eta\leqslant y}\\
        &=\prob{\ln{\xi}\leqslant y}\\
        &=\prob{\xi\leqslant e^y}\\
        &=F_{\xi}(e^y)\\
        &=\begin{cases}
            0,&e^y>1\\
            1-e^{-3y},&e^y\geqslant 1
        \end{cases}\\
        &=\begin{cases}
            0,&y<0\\
            1-e^{-3y},&y\geqslant 0
        \end{cases}
    \end{aligned}
\end{equation*}


\subsection*{b)}
Дано $f_{\eta}(y)=\begin{cases}
    0,&y<0\\
    3e^{-3y},&y\geqslant 0
\end{cases}\Longrightarrow \eta\sim \text{Exp}(\lambda=3)$

\section*{№8}
Имеется случайная величина $\xi\sim\text{Exp}(\lambda=1)$

\begin{equation*}
    f_{\xi}(x)=\begin{cases}
    0,&x<0\\
    e^{-x},&x\geqslant 0
\end{cases}, F_{\xi}(x)=\begin{cases}
    0,&x<0\\
    1-e^{-x},&x\geqslant0
\end{cases}
\end{equation*}

\ex Пусть $\eta=\lceil\xi\rceil$ — наименьшее целое число, большее, чем $\xi$

Найдем $\matwait{\eta}$. Так как $\xi$ — абсолютная непрерывная величина, то $\prob{\eta=0}=\prob{\xi=0}=0$

Пусть имеется $k=1,2,3,\ldots$, тогда 
\begin{equation*}
    \begin{aligned}
        \prob{\eta=k}&=\prob{\xi\in[k-1;k]}\\
        &=\mathbb{P}(\{\xi\leqslant k\}\setminus\{\xi\leqslant k-1\})\\
        &=\prob{\xi\leqslant k}-\prob{\xi\leqslant k-1}\\
        &=F_{\xi}(k)-F_{\xi}(k-1)\\
        &=\left[1-e^{-k}\right]-\left[1-e^{-(k-1)}\right]\\
        &=e^{-(k-1)}-e^{-k}\\
        &=e^{-(k-1)}\cdot\underbrace{(1-e^{e^{-1}})}_{p,q=1-p=e^{-1}}\\
        &=q^{k-1}p
    \end{aligned}
\end{equation*}
Значит, $\eta\sim \text{Geom}(p=1-e^{-1})\Longrightarrow\matwait{\eta}=\displaystyle\frac{1}{p}=\frac{1}{1-e^{-1}}$


























\end{document}