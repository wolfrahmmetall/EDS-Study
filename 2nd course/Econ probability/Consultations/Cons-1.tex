\documentclass[a4paper, 10pt]{article}
\usepackage{standart-header}
\fancyhead[R]{Винер Даниил. БЭАД232}
\fancyfoot [C] {\thepage}
\setlength{\parindent}{15pt}
\setlength{\parskip}{2mm}
\newcommand{\orr}{\text{ or }}
\newcommand{\andd}{\text{ and }}
\newcommand{\definition}{\textbf{Определение. }}
\fancyhead[L]{ТВиМС—1. Консультация—1 по минимуму 13.09.2024}

\begin{document}
\section*{№1}
\begin{enumerate}
    \item[$a)$]$P(A|B)=\displaystyle\frac{P(A\cap B)}{P(B)}=\frac{0.1}{0.4}=0.25$
    \item[$b)$] $P(A\cup B)=P(A)+P(B)-P(A\cap B)=0.3+0.4-0.1=0.6$
    \item[$c)$] $A$ и $B$ — независимы?

    \definition События $A$ и $B$ называются независимыми, если $P(A\cap B)=P(A)\cdot P(B)$

    \definition События $A$ и $B$ называются несовместными, если $A\cap B=\varnothing$

    Пусть $P(A)\ne0,P(B)\ne0$. Тогда, $A$ и $B$ несовместны, то $A$ и $B$ зависимы

    $0=P(A\cap B)=P(A)\cdot P(B)\ne0\Longrightarrow$ $A$ и $B$ зависимы
\end{enumerate}

\section*{№2}
\subsection*{Способ №1 (С помощью формулы умножения вероятностей)}
$P(A_1\cap \ldots \cap A_n)=P(A_1)\cdot P(A_2|A_1)\cdot P(A_3|A_1\cap A_2)\cdot\ldots\cdot P(A_n|A_1\cap\ldots\cap A_{n-1})$

Пусть имеются такие события: \begin{equation*}
    \begin{aligned}
        A_1&:=\{\text{первая буква — К}\}\\
        A_2&:=\{\text{вторая буква — О}\}\\
        A_3&:=\{\text{третья буква — Р}\}\\
        A_4&:=\{\text{четвертая буква — Т}\}
    \end{aligned}
\end{equation*}

Тогда, искомая вероятность:
\begin{equation*}
\begin{aligned}
    P(A_1\cap A_2\cap A_3\cap A_4)&=P(A_1)\cdot P(A_2|A_1)\cdot P(A_3|A_1\cap A_2)\cdot P(A_4|A_1\cap A_2\cap A_3)\\
    &=\frac{2}{13}\cdot\frac{2}{12}\cdot\frac{1}{11}\cdot\frac{1}{10}\\
    &=\frac{1}{4290}
    \end{aligned}
\end{equation*}

\subsection*{Способ №2 (комбинаторный)}
$P(A)=\displaystyle\frac{|A|}{|\Omega|},\ \Omega=\{(a_1,a_2,a_3,a_4):a_1\in L, a_2\in L, a_3\in L, a_4\in L, a_i\ne a_j\text{ при }i\ne j\}$

$|\Omega|=\displaystyle\frac{13!}{9!}$

$A=\{(K_1,O_1,P_1,T_1),(K_2,O_1,P_1,T_1),(K_1,O_2,P_1,T_1),(K_2,O_2,P_1,T_1)\}\longrightarrow$ 4 исхода

Индекс у букв означают какой по счету встретилась буква в слове <<КОМБИНАТОРИКА>>


\section*{№3}
\definition Совокупность событий $D_1,\ldots,D_n$ называется \textbf{разбиением} пространства элементарных событий, если 
$$\Omega=D_1\sqcup D_2\sqcup\ldots\sqcup D_n$$

$D_i:=\{\text{выбираем $i$-ю урну}\}$, где $i=1,2,3$ — разбиение $\Omega$

\begin{enumerate}
    \item[$a)$]\textbf{Формуа полной вероятности}
    $$P(A)=P(A|D_1)\cdot P(D_1)+\ldots+P(A|D_n)\cdot P(D_n)$$
    То есть, в нашем случае формула примет вид
    $$P(A)=P(A|D_1)\cdot P(D_1)+P(A|D_2)\cdot P(D_2)+P(A|D_3)\cdot P(D_3)$$
    $A:=\{\text{шар оказался белым}\}$
    \item[$b)$] $P(D_1|A)=\displaystyle\frac{P(A|D_1)\cdot P(D_1)}{P(A|D_1)P(D_1)+P(A|D_2)P(D_2)+P(A|D_3)P(D_3)}=\ldots$
\end{enumerate}

\section*{№4}
Обозначим сотрудников так:
\begin{equation*}
    \begin{aligned}
        D_1&:=\{\text{\textbf{опытный} сотрудник}\}\\
        D_2&:=\{\text{\textbf{неопытный} сотрудник}\}
    \end{aligned}
\end{equation*}

Пусть $A:=\{\text{совершена ошибка}\}$

Тогда, условия задачи можно записать так:
\begin{equation*}
    \begin{aligned}
        P(A|D_1)&=0.01\\
        P(A|D_2)&=0.1
    \end{aligned}
\end{equation*}

\begin{enumerate}
    \item[$a)$] $P(A)=P(A|D_1)\cdot P(D_1)+P(A|D_2)\cdot P(D_2)=0.01\cdot0.8+0.1\cdot0.2=0.028$
    \item[$b)$] $P(D_2|A)=\displaystyle\frac{P(A|D_2)\cdot P(D_2)}{P(A)}=0.714$

    Заметим, что события $(D_2|A)$ и $(D_1|A)$ образуют полную группу вероятностей, то есть $$P(D_2|A)+P(D_1|A)=1\Longrightarrow P(D_1|A)=0.286$$
\end{enumerate}












\end{document}